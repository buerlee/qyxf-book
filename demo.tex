% !TEX program = xelatex

\documentclass[
    b5paper,  % 默认为 a4paper
    %opensource, % 输出开源信息
    decoration,  % 打开装饰
]{qyxf-book}

\title{高等数学入门}
\subtitle{Advanced Mathematics Primer}  % 可选
\author{白山雁}
\date{2019 年 2 月 29 日}
%\typo{AlphaGo}  % 排版人员信息,选填

% 定制元信息
\org{\Large
    \textit{加里敦大学}\\
    \textsc{Caledon University}
}
\license{}

% 开启 opensource 选项时,以下信息必须填写
% \sourcepage{https://example.com/}

\begin{document}

\maketitle

\tableofcontents

\chapter{极限理论}
\section{极限的定义}

极限(limit)是一种过程……正如式 \eqref{eq:limit} 所言:

\begin{equation}\label{eq:limit}
\lim_{n\to\infty}\frac1{\sqrt n} = 0
\end{equation}

\section{极限的计算方法}
\subsection{观察法}

\begin{define}
    极限就是超越自我.
\end{define}

\begin{theorem}
    任何极限都可以直接观察得出.
\end{theorem}

\begin{lemma}
    以上内容,纯属扯淡.
\end{lemma}

\chapter{一元微分学}

% 以下为测试文本,请在编写时删除

微分学(differential calculus)是微积分的一部分,是通过导数和微分来研究曲线斜率、加速度、最大值和最小值的一门学科,也是探讨特定数量变化速率的学科.微分学是微积分的两个主要分支之一,另一个分支则是积分学,探讨曲线下的面积.

微分学主要研究的主题是函数的导数、相关的表示方式(例如微分)以及其应用.函数在特定点的导数可以说明函数在此输入值附近的变化率.寻找导数的过程即为微分.若以图示表示,函数在某一点的微分是函数图形在那一点的切线斜率(前提是在那一点的导数存在而且有定义).针对单实数变数的实值函数而言,函数在某一点的导数也就可以决定在那一点最佳的线性近似.微分和积分的关系可以由微积分基本定理来说明,此定理说明微分是积分的逆运算.

几乎所有量化的学科中都有微分的应用.例如在物理学中,运动物体其位移对时间的导数即为其速度,速度对时间的导数就是加速度、物体动量对时间的导数即为物体所受的力,重新整理后可以得到牛顿第二运动定律$F=ma$.化学反应的化学反应速率也是导数.在运筹学中,会透过导数决定在运输或是设计上最有效率的作法.

导数常用来找函数的极值.含有微分项的方程式称为微分方程,是自然现象描述的基础.微分以及其广义概念出现在许多数学领域中,例如复分析、泛函分析、微分几何、测度及抽象代数\footnote{摘自维基百科中文词条 - 微分学\cite{ref:wiki-diff-calculus}}.

\newpage

\section{导数的定义}

\begin{define}
	函数在某一点的\textbf{导数} $f'(x_0)=\lim\limits_{\Delta x \to 0}\dfrac{f(x_0 + \Delta x) - f(x_0)}{\Delta x}$.
\end{define}

\begin{table}[ht]
\centering
\caption{常用导数}
\begin{tabular}{c|c}
\toprule
\textbf{原函数} & \textbf{导函数} \\
\midrule
$C$      & $0$               \\
$x^\mu$  & $\mu x^{\mu - 1}$ \\
$e^x$    & $e^x$             \\
$\ln x$  & $\frac{1}{x}$     \\
$\sin x$ & $\cos x$          \\
$\cos x$ & $-\sin x$         \\
\bottomrule
\end{tabular}
\end{table}

\newpage

\section{L'Hôpital 法则}

\begin{theorem}
    \textbf{L'Hôpital 法则:} $\lim \dfrac{f(x)}{F(x)} = \lim \dfrac{f'(x)}{F'(x)}$.
\end{theorem}

\begin{thebibliography}{99}
\bibitem{texbook} KNUTH~D~E. The \TeX book [M]. Addison-Wesley: Massachusetts, 1986.
\bibitem{latex} 刘海洋. \LaTeX 入门 [M]. 人民邮电出版社: 北京, 2013.
\end{thebibliography}

\end{document}